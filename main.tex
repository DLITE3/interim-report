\documentclass[11pt,a4paper]{report}
\usepackage[margin=1in]{geometry}
\usepackage[japanese]{babel}
\usepackage{datetime2}
\usepackage{titlesec}
\usepackage{tocloft}
\usepackage{amsmath}
\usepackage{indentfirst}
\usepackage{fancyhdr}
\usepackage[dvipdfmx]{graphicx}

\newcommand{\AgendaBox}[2]{
    \large
    \textbf{#1}\\

    \vspace{0.2cm}

    \small 
    #2

    \vspace{0.5cm}
}
\newcommand{\NameBox}[2]{
    \small 
    #1\hspace{1cm}#2
}
% Chapter title formatting
\titleformat{\chapter}[hang]
  {\bfseries\huge} % Title style
  {第 \thechapter 章} % Label
  {1em} % Spacing between label and title
  {\huge} % Title font
% Section title formatting
\titleformat{\section}{\LARGE\bfseries}{\arabic{chapter}.\arabic{section}}{1em}{}
\titleformat{\subsection}{\large\bfseries}{\thesubsection}{1em}{}
\renewcommand{\thesubsection}{\normalsize\arabic{chapter}.\arabic{section}.\arabic{subsection}}

% ヘッダーとフッターをカスタマイズ
\fancyhead{} % 既存のヘッダーをクリア
\fancyfoot{} % 既存のフッターをクリア
\fancyfoot[C]{\thepage} % フッター中央にページ番号を表示

\begin{document}
\thispagestyle{empty}
\begin{center}
    \large
    \textbf{
      公立はこだて未来大学 2024年度 システム情報科学実習\\
      グループ報告書
    }\\

    \vspace{0.2cm}

    \small 
    \textbf{
      Future University Hakodate 2024 Systems Information Science Practice\\Group Report
    }

    \vspace{0.5cm}

    \AgendaBox{プロジェクト名}{境界なく人々の生活を支援する技術}
    \AgendaBox{Project Name}{DLITE3 : Technology that supports people's lives without boundaries}
    \AgendaBox{グループ名}{自然エンタメ班 班}
    \AgendaBox{Group Name}{Nature Entertainment Gropu}
    \AgendaBox{プロジェクト番号 / Project No.}{22}
    \AgendaBox{プロジェクトリーダ / Project Leader}{金子康一\hspace{1cm}Kaneko Koichi}
    \AgendaBox{グループリーダー / Group Leader}{
      \NameBox{未来太郎}{Mirai Taro}\\
    }
    \AgendaBox{グループメンバー / Group Member}{
      \NameBox{未来太郎}{Mirai Taro}\\
      \NameBox{未来太郎}{Mirai Taro}\\
      \NameBox{未来太郎}{Mirai Taro}\\
    }
    \AgendaBox{指導教員}{
      三上貞芳 伊藤精英 宮本エジソン正 島影圭佑
    }
    \AgendaBox{指導教員}{
      Mikami Sadayoshi Ito Kiyohide Miyamoto, Edson T. Shimakage Keisuke
    }
    \AgendaBox{提出日 / Date of Submission}{
      2024年7月12日 July 12, 2024
    }
    

\end{center}

% 空のページを挿入
\newpage
\thispagestyle{empty}
\mbox{}
\newpage

\clearpage
\pagenumbering{arabic}
\setcounter{page}{1}

{
    \centerline{
      \huge\textgt{概要}
    }
    \vspace{1cm}
    \noindent
     本プロジェクトでは、「視覚や聴覚に頼れない状況で役立つ装置の開発」をコンセプトとし、障がい者が抱える問題を当事者目線で
検討し、実用的な装置の開発に取り組んできた。頼れない感覚を別の手段で補うことで、不便を解消し、安全で快適な生活を支援す
ることを目指している。聴覚障がいや視覚障がい、色覚の障がい者を対象とした4つのグループに分かれ、それぞれ、特定の言葉や
音に反応するデバイス、画像の色をユニバーサルデザインに変換するアプリ、自力で避難することが難しい人のための補助デバイ
ス、障がい者が自然を楽しむためのデバイスの開発を行っている。

}
\newpage
{
    \centerline{
      \textbf{\huge\textgt{Abstract}}
    }
    \vspace{1cm}
    \noindent
    \space Under the concept of "developing devices that are useful in situations where one cannot rely on sight or hearing," this project examines the
problems faced by people with disabilities from the perspective of the people concerned, to develop practical devices. By supplementing
unreliable senses with other means, the project aims to eliminate inconvenience and support safe and comfortable living. The project is
divided into four team targeting people with hearing disabilities, visual disabilities, and color blindness. Each team is developing devices that
respond to specific words and sounds, applications that convert the color of images to universal design, assistive devices for people who
have difficulty evacuating on their own, and devices that allow people with disabilities to enjoy nature.

}
\newpage

% Table of contents (optional)
\tableofcontents
\newpage

% 本文ページ
\chapter{はじめに}
\section{背景}
\noindent
hogehoge
\section{先行研究}
\noindent
hogehoge
\section{研究動機}
\noindent
hogehoge
\section{目的}
\noindent
hogehoge

\chapter{関連研究}
\section{使用技術}
\noindent
hogehoge
\section{解決手法}
\subsection{サブセクション1}
\noindent
hogehoge
\subsection{サブセクション2}
\noindent
hogehoge

\chapter{活動の要約}
\section{成果}
\noindent
hogehoge
\section{活動計画}
\noindent
hogehoge

\end{document}